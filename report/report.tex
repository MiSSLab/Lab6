\documentclass[a4paper,10pt]{article}
\usepackage{amsmath}
\usepackage{amssymb}
\usepackage{pifont}
\usepackage[utf8]{inputenc}
\usepackage[left=2.5cm,right=2.5cm,top=2.5cm,bottom=2.0cm]{geometry}
\usepackage{multirow}
\usepackage{geometry}
\usepackage{pdflscape}
\usepackage{listings}

\lstset{basicstyle=\footnotesize\ttfamily,breaklines=true}
\lstset{framextopmargin=60pt,frame=all}
\title{Modelling and Simulation of Systems\\ \Large
Exercise 6: Visualization of a Simulation}
\author{Agata Radys, Paweł Cejrowski, Łukasz Myśliński}
\date{\today}

\pdfinfo{%
  /Title    (Modelling and Simulation of Systems - Exercise 6: Visualization of a Simulation)
  /Author   (Agata Radys, Paweł Cejrowski, Łukasz Myśliński)
  /Subject  (Modelling and Simulation of Systems)
  /Keywords (simulation; vizualization; dla; diffusion-limited aggregation; modelling and simulation of systems)
}
\begin{document}
\newgeometry{margin=1.8cm}
\maketitle

\section*{Phenomenon description}
The modelled phenomenon is the temperature propagation in a fluid, based on Brownian motion. It occures in the gases and
liquids. Can be observed in floor heating in an apartment.
\section*{Model}
The phenomenon will be modelled using diffusion limited aggregation (DLA) process. In this process particles are undergoing
random walk due to Brownian motion and cluster together to form aggregats of such particles. Particle is colored based on a time
when it aggregated to the cluster.
\\
We model phenomenon in 2D. First, seed of "heated" particles is set on the floor (the lowest row). Then, particles are created
10 points from the top/ceiling. And take random walk with probabilites of 0.25 to go right or left, 0.15 to go up and 0.35 to go down.
Going down is most probable due to gravity force. When a particle has in its neighbourhood one heated particle, it becomes heated as well.
Algorithm stops when any heated particle is almost under the ceiling (above starting point). Algorithm is presented in the following pseudocode.

\begin{lstlisting}[]

N // modelled space width/height
dla = matrix[N][N]
particles = 0
set the bottom row as a seed
launch = N - 10 // all particles start 10 pixels down from the top
done = false

while (!done):
    x = random(0,N)
    y = launch;

    // random walk
    while (particle(x,y) is within the considered 2D space):

        r = random(0,1);
        if (0    < r < 0.25) x--
        if (0.25 < r < 0.50) x++
        if (0.50 < r < 0.65) y++
        if (0.65 < r < 1) y-- // it is the biggest due to gravity force

        if (neighbour of [x,y] is occupied):
            dla[x][y] = true
            particles++
            printPixel(x, N - y - 1, assignColor(particles))

            if (y > launch):
                done = true

            break // particle random walk finishes when it found neighbour
\end{lstlisting}



\end{document}

