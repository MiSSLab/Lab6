\documentclass[a4paper,10pt]{article}
\usepackage{amsmath}
\usepackage{amssymb}
\usepackage{pifont}
\usepackage[utf8]{inputenc}
\usepackage[left=2.5cm,right=2.5cm,top=2.5cm,bottom=2.0cm]{geometry}
\usepackage{multirow}
\usepackage{geometry}
\usepackage{pdflscape}
\usepackage{listings}

\lstset{basicstyle=\footnotesize\ttfamily,breaklines=true}
\lstset{framextopmargin=50pt,frame=bottomline}
\title{Modelling and Simulation of Systems\\ \Large
Exercise 6: Visualization of a Simulation}
\author{Agata Radys, Paweł Cejrowski, Łukasz Myśliński}
\date{\today}

\pdfinfo{%
  /Title    (Modelling and Simulation of Systems - Exercise 6: Visualization of a Simulation)
  /Author   (Agata Radys, Paweł Cejrowski, Łukasz Myśliński)
  /Subject  (Modelling and Simulation of Systems)
  /Keywords (simulation; vizualization; dla; diffusion-limited aggregation; modelling and simulation of systems)
}
\begin{document}
\newgeometry{margin=1.8cm}
\maketitle

\section*{Phenomenon description}
The modelled phenomenon is the temperature propagation in a fluid, based on Brownian motion. It occures in the gases and
liquids. Can be observed in floor heating in an apartment.
\section*{Model}
The phenomenon will be modelled using diffusion limited aggregation (DLA) process. In this process particles are undergoing
random walk due to Brownian motion and cluster together to form aggregats of such particles. Particle is colored after a time
when it aggregated to the cluster.
\\
We model phenomenon in 2D.


\begin{lstlisting}[language=java,caption={Algorithm for generating consecutive steps}]

N - modelled space width and height
dla = matrix[N][N]
particles = 0
set the bottom row as a seed
launch = N - 10 // all particles start 10 pixels down from the top

done = false
while(!done):
    x = random(0,N)
    y = launch;
    // random walk
    while (is within the considered 2D space):
        r = random(0,1);
        if (0    < r < 0.25) x--
        if (0.25 < r < 0.50) x++
        if (0.50 < r < 0.65) y++
        if (0.65 < r < 1) y-- // it is the biggest due to gravity force

        if (neighbour of [x,y] is occupied) {
            dla[x][y] = true
            particles++
            printPixel(x, N - y - 1, assignColor(particles))

            if (y > launch):
                done = true
            break // paricle random walk finishes when it is marked
\end{lstlisting}



\end{document}

